% Introduction and/or Motivation
\section{Introduction}
\label{sec:intro}
The rise in popularity of sharing systems has increased the bike-sharing system to gain popularity in urban areas. The concept of bike sharing systems was introduced in 1965 Amsterdam. Since the introduction of bike-sharing systems, much research has been dedicated to different aspects of these systems. The benefits of bikes in urban areas, where travel distance is short, and the parking prices are high, has caused the demand for bike-sharing systems to increase. Some of the studies related to this system are descriptive and mine the data to get a better understanding of the operations. 

Data Visualization and analysis of the current operations can assist in getting a better grasp of the system. Data visualization, which is defined as the effort of placing data in a visual context, can assist in better understanding the problem. When we have many data points, visualization of the data becomes challenging. The purpose of this study is to analyze the bike-sharing system to group the bike renting trends season-wise, locate the busiest stations in a locality, analyze the revenue generated, costs incurred in maintaining the bikes and number of monthly or flex passes taken by the customers every quarter. We mine the Metro Bike Share, Los Angeles data and discuss the findings of this data set. We examine the bike-share system during different quarters of the year and during both peak and non-peak hours. Therefore potentially analyzing the sustainability of the bike sharing system. This study also helps in gaining a better understanding of the urban mobility of Los Angeles residents. 

% You may want to end this section by summarizing it again as a list:
The aims of this research are:
\begin{itemize}
  \item A map visualization to locate the busiest station in a locality
  \item A visualization to analyze the bike renting trends season wise
  \item a visualization to indicate the revenue generated and costs incurred 
  \item a visualization to track the number of monthly or flex passes issued to the customers every quarter
\end{itemize}
