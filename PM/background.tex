\section{Background}
\label{sec:background}

Our goal is to make visualizations with the data acquired from kaggle and bikeshare.metro.net which will give us insights about various trends prevalent in this industry. 
As the data table is clearly demarked with fields like duration, start time, end time etc., it does not beg much effort from the reader to glean information from the data. 
The duration field marks the time of travel each customer takes for a particular trip. 
For the accurate representation of the start and end location of each journey, the geographic longitude and latitude have been represented as start_lon, start_lat, and end_lon, end_lat respectively. 
There are three types of passes available for the customers. They are Walk-Up, Monthly Pass and Flex-Pass, which give the customers access to the bikes for 1 day, 30 days and 365 days respectively.  

\subsection{Related Work}
\label{sec:related}

We took inspirations from the work of Zamir, Shafahi, Haghani(2017) ~\cite{Zamir, Shafahi, Haghani:2017: University of Maryland} which delves into the understanding and visualization of the same kind of data from bike sharing systems in the district of Columbia. 
Moreover, Heinz(2017)~\cite{Heinz:2017: NYC Data Science}, which discusses Bay area bike share data and Patterson(2017) ~\cite{Patterson:2017: Urbandatacyclist.wordpress.com} helped a lot in determining the trends and patterns to look for. 
