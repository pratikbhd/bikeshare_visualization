\label{sec:overview}
Bike share schemes are an increasingly prevalent mode of intra-city transportation. The concept of bike sharing systems was introduced in 1965 Amsterdam. A bike sharing program is a system of supplying bikes for hire for point-to-point transportation. This program enables convenient active transportation as people have an option to ride between two stations in a defined geographical area. Since the introduction of bike-sharing systems, much research has been dedicated to different aspects of these systems. The benefits of bikes in urban areas, where travel distance is short, and the parking prices are high, has caused the demand for bike-sharing systems to increase. Bike share systems may help mitigate automobile congestion and reduce pollution, although relatively little research has been done to asses their actual impact in these areas. Benefits to users include potentially reduced commute times by perhaps as much as 10\% \cite{Sakari:2013:Data} and a healthier lifestyle to lead. 
 
Besides visualizing bike sharing systems as a new means of public transportation, such community shared programs offer a new way to look into the dynamics of movements inside a city, and more generally into its activity. In a sense, it provides digital footprints that reveal the activity of people in the city over time and space and makes possible their analysis. Different issues motivate the study of such a system. Some questions are about the usage patterns of this kind of transport, with reference to social or economic studies of transportation, while others are about the system itself: does the service work correctly? Can it be optimized? Can one regulate the availability of bikes? Some of the studies related to this system are descriptive and mine the data to get a better understanding of the operations. 

Data Visualization and analysis of the current operations can assist in getting a better grasp on the performance of the system. Data visualization, which is defined as the effort of placing data in a visual context, can assist in better understanding the problem. When we have many data points, the visualization of the data becomes challenging. The purpose of this study is to analyze the bike-sharing system by applying filters like renting trends season-wise, locate the busiest stations in a locality and number of monthly or flex pass taken by the customers for different time ranges. We mine the Metro Bike Share, Los Angeles data and discuss the findings of this data set. We examine the bike-share system during different quarters of the year. Therefore potentially analyzing the sustainability of the bike sharing system. This study also helps in gaining a better understanding of the urban mobility of Los Angeles residents. In terms of visualizations, our project does not develop a novel visualization but uses prevalent methods of visualizing data to try and find trends of a certain behavior from this dataset. We analyze various types of visualizations and bring together several ideas to form an interactive and easy-to-use interface that helps users navigate through the bike-sharing data in a more ordered and systematic way. Our choice of visualization holds ground with the design practices that make a good visualization and also take into consideration some design techniques that enhance the user experience further.

The work that was completed for this milestone are:
\begin{itemize}
  \item We have mainly worked on the refinements suggested from project milestone 1.
  \item We have decided on a visualization that has a map along with the time chart histogram. The changes in time data will change the map visualization based on different parameters. 
  \item We analyzed more research papers and tools to not only learn about past work in this field but also get sufficient understanding of what entails a good and robust visualization for a data of this type and structure.
  \item We researched more into possible evaluation techniques that might suit our project goals and selected the one that best fits our tasks and objectives.
  \item We have refined our evaluation section in more depth.
\end{itemize}
