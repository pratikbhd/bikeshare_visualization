\section{Lessons Learned} 
\label{sec:lessons_learned}

In the duration of developing this visualization and analyzing the data for visualization we have learned few noticeable things as follows:

\subsection{What lessons we learned that could possibly be transferred to other visualization problems and contexts?}

\begin{itemize}
\item The most important lesson we have learned in designing our visualization is to decide how the visualization shouldn't look and how visualization should look. We have had many brain storming discussions in this area with the team members.
\item A user-centered design approach can result in an intuitive interface, which, in turn, can open up new possibilities. In our visualization, the appropriate user interface can open new possibilities by providing a big picture of operation and information within the metro bike share system.
\item Our visualization provides a customized display of information. It shows the graphical representations of bike renting trends in Los Angleles area. They also can measure efficiencies and inefficiencies, quickly identify correlations, and illustrate trends. All of these features combine to help in making better-informed decisions about the sustainablity of the bike sharing system.
\item One challenge for our visualization is that the interpretation of data can be difficult. Another issue is that the greater transparency provided may not always be desirable. There is also a danger of accidentally filtering by the wrong information. 
\item The most important practice to follow when designing a visualization is to find out who the information consumers are and whether they are managing data, consuming reports, or both. 
\item We have to design the visualization around what the users need at their fingertips to analyze the data. To this end, get the requirements from users. Research what they need and want to see. Find out which searches they run the most often and which ones they should be running to analyze the trends in the data.
\item Use a whiteboard layout and requirements to develop standard visualization. Whenever possible, have the interactive parts perform more than one function, and allow interactive parts to be clickable for more information. Clicking a bar in the histogram, for example, could display more current bike renting numbers. When deciding on the appropriate interactions, be careful to select the interaction type that best conveys the information being displayed. 
\item The data can be complex. Mistakes like graphically representing non-quantitative data, can lead to confusing results. Other common mistakes include mixing and matching unrelated data and starting the visualization with a blank screen instead of showing important and relevant information. We need to show information that is usable to the viewer in some manner.
\item A thorough requirements analysis is part of designing a visualization. The design process is to build, evaluate, test, and refine prototypes.
\item Data may be missing, incomplete, inaccurate, incorrectly entered, inconsistently coded, diversely coded, duplicated, or use different standards. We need to cleanse the data properly before we start the visualization.
\item  The underlying technology can be important. If you have a system that uses a graphic that uses Flash, be aware that it won’t run on the iPad. You’ve got to keep in mind, that over time, the visualization needs to change or evolve to run on many different technologies.
\item Focus on identifying and understanding the problem you’ll be solving. You’ll never be able to solve every problem and overcome every data use barrier in one project. Work with your users to develop a specific focus and thoroughly understand the barriers and challenges from their perspectives so you can tackle the most pressing issues.
\item Engage the right people, as we need a diverse range of perspectives and experiences to uncover problems and co-create solutions. 

\end{itemize}

